\chapter{Introducción. Antecedentes} \label{ch:introduction}

\textit{En esta sección se presenta una breve introducción al tema y al estado del arte. La extensión no debe superar las 1000 palabras.
Se pueden introducir referencias bibliográficas (detalladas en la sección correspondiente).}

La firma espectral de un cuerpo es el tipo de repsuesta que el mismo posee cuando es radiado. Cada tipo de superficie interactúa con la radiación de manera diferente, absorbiendo unas longitudes de onda muy concretas y reflejando otras diferentes en unas proporciones determinadas. Actualmente la misma se mide utilizando satélites activos, transmitiendo un tipo de señal conocida, o pasivos, donde el transmisor es u otro satélite o el sol. 

Dado que al utilizar al sol como transmisor se desconoce la fase inicial de la señal transmitida, no se puede determinar el desfase que el cuerpo induce sobre la señal. Por lo tanto, se requiere la utilización de un radar activo, el cual, generalmente es un radar pulsado, lo cual implica que la transmisión y recepción se realizan en tiempos distintos.

Como la interacción entre la superficie y la señal varía entre polarizaciones verticales u horizontales, se requiere que el radar pueda transmitir y recibir en ambas polarizaciones para poder caracterizar completamente la respuesta del cuerpo iluminado. Si los datos se ordenan en todas las combinaciones posibles de polarizaciones en forma de una matriz, la misma se llama matriz de dispersión.

Para poder medir la matriz de retrodispersión es necesario tener caracterizado todos los subsistemas del radar, así como conocer la distancia entre el mismo y el cuerpo, para poder eliminar el efecto que el medio induce en la señal. 

En este trabajo se construye un modelo de ingeniería de un radar para la determinación de la matriz de dispersión de los cuerpos iluminados. Como la distancia entre el radar y el cuerpo es chica, entre de 2 a 4 metros, la utilización de un radar pulsado no es viable dado que el tiempo de ida y vuelta de la señal resulta comparable con la longitud del pulso transmitido. Es por eso que se uitliza un esquema de frecuencia modulada de transmisión contínua (radar FMCW), lo cual implica que la transmisión y recepción es simultánea. Este tipo de radares son utilizados para diversas aplicaciones como altímetros, medidores de velocidad de vehículos, radares de proximidad para evitar choques o para imágenes SAR \cite{Richards2010}.

Este tipo de radares mide la diferencia de fase entre la señal transmitida y recibida para determinar la distancia y velocidad del cuerpo iluminado. Es por eso que es necesario alún tipo de modulación de la señal transmitida. Las modulaciones mayormente utilizadas son triangular, diente de sierra, cuadrada o escalonada, cada una posee diferentes propiedades \cite{Basics2015}.

Como no se encontró ninguna publicación que utilicen este tipo de radares para el área de interés al cual se lo quiere aplicar, por lo cual todas las publicaciones que se utilizan como referencia no aplican completamente sino que se toma parte de dicha información para el interés de este trabajo. 

\begin{itemize}

    \item El procesamiento de la señal recibida con distintos tipos de modulación triangular \cite{Chang2006}, modulación escalonada \cite{steppedFreq}, con diente de sierra \cite{Shen, Varavin2007a}. 

    \item En \cite{Brennan2014a} se hace uso de un radar FMCW para medir distancias con precisión del $\si{\milli\meter}$ haciendo un análisis de la potencia recibida y del desfase requerido que haya en la resolución de distancia para evitar ambigüedades en rango.

    \item En \cite{Shen} se realiza un análisis de la complejidad numérica y algorítmica del demodulador transmitiendo con una modulación de diente de sierra.

    \item En \cite{Kurt2007} se presentan distintos mecanismos para mejorar la resolución en rango utilizando una modulación triangular. 

    \item En \cite{Lipa1990} se expone el procesamiento de un radar FMCW para detección de blancos en movimiento.

    \item En \cite{Brooker2005} se detalla sobre el diseño e implementación de un radar FMCW enfocándose en ruido de fase del generador, de alinealidades del modulador, su relación con el ancho de banda y con la resolución en rango. Otras publicaciones que detallan un diseño de radar son \cite{Chan2009, Wavemaker2015}. 

\end{itemize}

