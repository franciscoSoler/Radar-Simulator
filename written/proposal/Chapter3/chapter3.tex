\chapter{Desarrollo previsto de la tesis} \label{ch:development}

\textit{En esta sección se presenta un resumen del desarrollo del trabajo}

El trabajo comienza con la búsqueda bibliográfica en publicaciones especializadas en los distintos ámbitos del tema de Tesis. Dicha información será utilizada para iniciar dos tareas paralelas. En primer lugar, se irá trabajando en el diseño y construcción de los distintos subsistemas que componen el radar, ya sea la cadena de transmisión y recepción de la señal de radio frecuencia como la parte de baja frecuencia. En segundo lugar, se implementará el simulador del esquema completo de modulación, transmisión, recepción y procesamiento de la señal recibida para obtener el desfase y atenuación que el cuerpo simulado induce en la misma.

A su vez, el simulador será utilizado para desarrollar y validar al procesador de señales del radar real. El cual será utilizado para realizar las mediciones reales de un corner reflector o un cuerpo similar a diferentes distancias, y de esta forma obtener su matriz de dispersión. Para ello se realizan las siguientes hipótesis de base:

\begin{itemize}
    \item Todos los elementos del sistema son lineales e invariantes en el tiempo.
    \item El medio de propagación es homogéneo.
    \item El ruido posee una distribución gaussiana.
    \item El comportamiento de los distintos componentes no se modifica al variar su temperatura.
\end{itemize}

Una vez construido el radar, se procederá a medir cada uno de sus subsistemas para evitar cualquier tipo de problemas entre sus interfaces así como para caracterizarlos en el rango de frecuencia de funcionamiento. Se utilizarán un osciloscopio para medir la forma de onda de la señal del modulador y un analizador de redes vectorial para obtener los parámetros S de las antenas, divisores de potencias, atenuadores y conectores. A su vez, se utilizará un analizador de espectro para obtener la potencia generada por el generador, la transmitida por el radar, su ancho de banda y el diagrama de radiación. Por último, con un generador y un osciloscopio se caracteriza la respuesta del filtro pasa bajos. 

Los resultados de estas mediciones son utilizados en el receptor del radar para disminuir el error sistemático al realizar las mediciones de la matriz de dispersión del cuerpo iluminado. Es importante que el radar esté dentro de una cámara anecoide o en un ambiente libre de otros cuerpos a la hora de medir el diagrama de radiación, los parámetros S de las antenas y durante las realizaciones de las mediciones con el radar. En caso de no contar con una, dichas mediciones serán deterioradas dado a que se recibirán ecos de otros cuerpos.

A modo experimental se programará al receptor del radar para suprimir el clutter del ambiente. La misma consistirá en realizar dos mediciones consecutivas, primero sin el cuerpo a ser iluminado y luego con el mismo. Por último, en base a todas las mediciones, tanto del radar como de otros cuerpos, se definirá la eventual necesidad de efectuar experiencias adicionales o modificaciones al radar.

A continuación se presenta el cronograma de actividades propuesto.

\begin{table}[htb]
  \caption{Cronograma de actividades.}
  \centering
  \label{tab:antennasParameters}
  \begin{tabular}{l *{12}{c}}
  \toprule
  \multirow{2}{*}{\textbf{Tareas}} & \multicolumn{12}{c}{\textbf{Meses}} \tabularnewline
  \cmidrule{2-13}
   & 1 & 2 & 3 & 4 & 5 & 6 & 7 & 8 & 9 & 10 & 11 & 12 \tabularnewline
  \midrule

  Búsqueda de la bibliografía & X & X & & & & & & & & & & \tabularnewline

  Análisis numérico, señal recibida & & X & X & & & & & & & & & \tabularnewline

  Implementación modelo simulador & & & X & X & & & & & & & & \tabularnewline

  Diseño y construcción del radar & & & X & X & X & X & X & & & & & \tabularnewline

  Mediciones de subsistemas del radar & & & & & & X & X & & & & & \tabularnewline

  Implementación procesador señal  & & & & & & & X & X & X & X & & \tabularnewline

  Mediciones de aplicación & & & & & & & & & & X & & \tabularnewline

  Escritura de la tesis & & & & & & & & & & & X & X \tabularnewline

  \bottomrule
  \end{tabular}
\end{table}
