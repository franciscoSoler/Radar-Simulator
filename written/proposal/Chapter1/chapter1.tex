\chapter{Objeto y área de la tesis} \label{ch:object}

\mynote{En esta sección se presenta un resumen de no más de 200 palabras con el objetivo de la Tesis a desarrollar, indicando
explícitamente los aportes creativos y/o novedosos del trabajo. Área profesional de relevancia: señalar el área profesional en que
se encuadra el tema y desarrollo de la Tesis dentro de las incumbencias del título de Ingeniero Electrónico de la UBA.}

El presente trabajo de tesis responde a la motivación de modelizar, construir, medir y validar un prototipo de ingeniería de un radar de frecuencia modulada de transmisión continua (FMCW) de modo tal que sirva como prueba de concepto para un desarrollo a futuro con instrumental más preciso para la medición de la matriz de dispersión de distintos cuerpos iluminados. El principal objetivo de las mediciones realizadas con dicho radar es el de determinar cuáles son las fuentes principales de incertidumbre a la hora de caracterizar el objeto iluminado. Es importante destacar que no se registran experiencias de uso de un radar FMCW para este tipo de mediciones.

El tema propuesto involucra poner en acción los conocimientos adquiridos a lo largo de la carrera Ingeniería Electrónica en lo que respecta al diseño, simulación, construcción y medición de circuitos electrónicos y de antenas. A su vez, respecta a generación, transmisión, recepción, registro, y procesamiento de señales electromagnéticas de variadas frecuencias.
