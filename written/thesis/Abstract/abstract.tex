% ************************** Thesis Abstract *****************************
% Use `abstract' as an option in the document class to print only the titlepage and the abstract.
\begin{abstract}
Hoy en día los radares FMCW son utilizados en innumerables aplicaciones: altímetros para medir la altura al descender un avión, o por ejemplo medidores de velocidad de vehículos y radares de proximidad para evitar choques o para imágenes SAR.

En este trabajo se ha estudiado, especificado, modelizado y construído un modelo de ingeniería a modo de prueba de concepto o desarrollo preliminar para identificar las mayores problemáticas y fuentes de incertidumbre presentes a la hora de determinar los parámetros de dispersión de un blanco estático a diferentes distancias.

Para la validación y verificación del sistema en su totalidad se ha implementado un simulador, el cual modeliza cada subsistema y se le pueden agregar individualmente distintos tipos de incertidumbre para detectar sus efectos en el  sistema en su conjunto.

Se presentan todas las caracterizaciones necesarias sobre los distintos subsistemas junto a los bancos de trabajo y criterios adoptados a la hora de elegir las características necesarias de las diferentes posibles tecnologías.

Finalmente, se han presentado mediciones de aplicación realizadas utilizando un gabinete metálico y un reflector especialmente construído como blancos. En base a lo anteriormente mencionado se han obtenido conclusiones y se han propuesto diversas líneas futuras a seguir.
\end{abstract}
