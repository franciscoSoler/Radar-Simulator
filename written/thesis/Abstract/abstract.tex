% ************************** Thesis Abstract *****************************
% Use `abstract' as an option in the document class to print only the titlepage and the abstract.
\begin{abstract}
Hoy en día los radares FMCW son totalmente utilizados para innumerables aplicaciones como altímetros para medir la altura exacta al descender un avión, como medidores de velocidad de vehículos y radares de proximidad para evitar choques o para imágenes SAR.

En este trabajo se estudia y construye un modelo de ingeniería como prueba de concepto para identificar las mayores problemáticas y fuentes de incertidumbre presentes a la hora de determinar los parámetros de dispersión de un blanco estático a diferentes distancias.

Para la validación y verificación del sistema en su totalidad se implementó un simulador, el cual modeliza cada subsistema y se le puede agregar individualmente distintos tipos de incertidumbre para detectar sus efectos en el  sistema en su totalidad.

Por último se presentan todas las caracterizaciones necesarias sobre los distintos subsistemas junto a los bancos de trabajo y criterios adoptados a la hora de elegir las características necesarias de las diferentes posibles tecnologías.

Por último, se realizaron mediciones de aplicación utilizando una valija metálica como blanco. En base a lo anteriormente mencionado se obtuvieron conclusiones y se detectaron diversas líneas futuras a seguir.
\end{abstract}
