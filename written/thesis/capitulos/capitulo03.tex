%Capítulo 3

\chapter{Procesamiento de la señal de un radar FMCW}
\label{ch:classicalCalibration}
\lhead{\emph{Procesamiento de la señal de un radar FMCW}}
%----------------------------------------------------------------------------------------




Hay distintos tipos de señales con las que se puede transmitir utilizando dicho tipo de radar, las cuales están resumidas en 
el cuadro \ref{tab:signalTypes}.
\todo[inline]{completar la tabla}
\begin{table}[H]
  \footnotesize
  \centering
  \begin{tabular}{|c|c|}
	\hline
	\textbf{Tipo de señal} & \textbf{Característica} \tabularnewline \hline 
	triangular &  1275000000 [Hz] \tabularnewline\hline 
	diente de sierra & 0 [dB] \tabularnewline \hline 
	cuadrada & 0 [deg] \tabularnewline \hline 
	desiredTxPower & 6 [dB] \tabularnewline \hline 
  \end{tabular}
  \caption{Configuración de la antena común para todos los ensayos.}
  \label{tab:signalTypes}
\end{table}

En esta tesis se utiliza el diente de sierra en frecuencia. La figura \ref{fig:sawtoothSignal} ilustra para un caso general los parámetros 
que definen dicha señal, los cuales son el ancho de banda de la señal transmitida ($B$) y el tiempo de repetición de dicho 
pulso ($PRT$).

\todo[inline]{hacer una firgura doble donde muestra la repeticion de chirp y la señal en frecuencia}
\begin{figure}[H]
 \centering
 \includegraphics[width=10cm]{gfx/classic_cal_scheme.png}
 \caption{señal diente de sierra en frecuencia.} 
 \label{fig:sawtoothSignal}
\end{figure}

La ecuación que define cada repetición del pulso está descripta a continuación:
\begin{equation}
	f(t) = f_0 + \dfrac{B}{T}(t-\dfrac{T}{2}),\quad 0 \le t < T
	\label{eq:signalFrequency}
\end{equation}

Dado que la fase de una señal es la integral de su frecuencia sumada a una constante, se obtiene la fase instantánea transmitida a partir de
la ecuación \ref{eq:signalFrequency},
\begin{equation}
	\varphi_t(t) = 2\pi f_0t + \dfrac{2\pi B}{2T}(t^2-Tt) + \phi_0,\quad 0 \le t < T
	\label{eq:signalFrequency}
\end{equation}

El eco recibido por un objeto puntual en un rango $R$ y defasando la señal en $\theta$ llega al radar con la fase, 
\begin{equation}
	\varphi_r(t) = w_0(t-\tau) + \dfrac{2\pi B}{2T}((t - \tau)^2-T(t - \tau)) + \theta + \phi_0,\quad 0 \le t < T
	\label{eq:signalFrequency}
\end{equation}

\todo{buscar la e del error}
Donde $\tau=2R\sqrt{e_r}/c$ es el tiempo ida y vuelta de la señal entre el transmisor y receptor del radar.
\todo[inline]{buscar bibliografia del mixer dado que faltan todos los armónicos}
Luego, la señal recibida se la multiplica con la transmitida, por identidad trigonométrica, se obtiene 
\begin{equation}
	x(t) = \dfrac{A_tA_r}{2}(\cos(\varphi_t(t)+\varphi_r(t)) - \cos(\varphi_t(t)- \varphi_r(r)))
	\label{eq:signalFrequency}
\end{equation}

Donde $A_r$ es la amplitud de la señal recibida, la cual es modificada por el ida y vuelta del medio y por el blanco donde se 
genera el eco de la señal.
Que, luego del pasabajos, el término donde está la suma de fases posee una frecuencia de $2w_0$ queda completamente atenuado, 
quedando solamente el término donde está la resta de fases,
\begin{equation}
	x_d(t) = \dfrac{A_tA_r}{2}\cos(w_0\tau + \dfrac{2\pi B\tau}{T}(t - \dfrac{T}{2}) - \dfrac{2\pi B\tau^2}{2T} - \theta)
	\label{eq:signalFrequency}
\end{equation}


Es importante notar que la fase inicial de transmisión se elimina, por lo tanto no importa cual es la fase inicial de cada pulso
transmitido, tampoco importa si varía pulso a pulso dado que se elimina al realizar la mezcla.

Del resultado de la ecuación \ref{eq:signalFrequency} se observa que la fase está compuesta por cuatro términos.
\begin{itemize}
	\item $w_0\tau$ es el término utilizado para determinar distancias con alta precisión en blancos conocidos. \todo{poner 
		referenciaPhase-sensitive FMCW radar system for high-precision Antarctic ice shelf profile monitoring}
	\item $\dfrac{2\pi B\tau}{T}(t - \dfrac{T}{2})$ es un término de fase lineal con el tiempo que representa la frecuencia de la 
		señal.
	\item $\dfrac{2\pi B\tau^2}{2T}$ es un offset en la señal, usualmente pequeño.
	\item $\theta$ es el término de fase originada por el blanco donde incidió la señal transmitida. 
\end{itemize}

Derivando la fase obtenida en la ecuación \ref{eq:signalFrequency} y utilizando la definición de $\tau$, se obtiene la 
frecuencia de dicha señal:
\begin{equation}
	w_d = \dfrac{2\pi B\tau}{T} \rightarrow f_d = \dfrac{2BR\sqrt{e_r}}{cT}
\end{equation}


\begin{comment}
\section{Calibración clásica}

Como fue introducido previamente, una antena es alimentada por un generador y consta de una RFDN y un panel de elementos 
radiantes. Como pueden haber dispersiones en el comportamiento de cada componente, se hace uso de la calibración para 
compensarlas. En particular, este método se lo utiliza para poder medir, detectar y corregir el mal funcionamiento de parte de 
la RFDN, en particular todas las desviaciones se las atribuyen los módulos de Transmisión/Recepción. A su vez, también se 
puede detectar si uno de dichos módulos queda inhabilitado a causa que cierta parte de la cadena de transmisión/recepción 
o dicho componente se destruyó.

Una de las principales limitaciones, esta calibración interna no puede medir las partes pasivas del sistema que están fuera del
lazo de calibración interna (por ejemplo los módulos radiantes), ni la ganancia absoluta, debido a la ausencia de un blanco 
estándar de calibración \cite{Wang2010}.

En la figura \ref{fig:classic_cal_scheme} se muestra el esquema de calibración, en el cual se observan tres modos de
calibración. Cada uno posee distintos caminos, \textbf{P1} (líneas rojas) caracteriza el camino de transmisión, \textbf{P2}
(línea azul) caracteriza el camino de recepción y \textbf{P3} la electrónica central (CE) junto a los puertos auxiliares de
transmisión/recepción. \textbf{P3} es utilizado para corregir posibles variaciones en los pulsos \textbf{P1} y \textbf{P2}
\cite{Makhoul2012}. Se puede apreciar que, para este esquema en praticular, tanto los elementos radiantes como sus cables de 
interconexión quedan fuera de los lazos de calibración.

\begin{figure}[H]
 \centering
 \includegraphics[width=10cm]{gfx/classic_cal_scheme.png}
 \caption{Esquema de calibración interna: camino de calibración de pulsos \textbf{P1} (Tx) en rojo, \textbf{P2} (Rx) en azul
 y \textbf{P3} (electrónica central) en verde \cite{Makhoul2012}.}
 \label{fig:classic_cal_scheme}
\end{figure}

Como ejemplos de satélites que utilizaron dicho método, se los puede nombrar al E-ERS-1, el SIR-C \cite{Curlander1991}, el
terraSAR-X \cite{Schwerdt2005} y el ENVISAT ASAR \cite{Loop}. La figura \ref{fig:calibrMethods} muestra los esquemas
de calibración de los primeros dos sistemas mencionados.

\begin{figure}[H]
	\centering
	\subfloat[]{
		\includegraphics[width=7cm]{gfx/sirCalibration.png}}
 	\subfloat[]{
		\includegraphics[width=7cm]{gfx/eersCalibration.png}}
	\caption{Esquemas de calibración del SIR-C y E-ERS-1 \cite{Curlander1991}.}
	\label{fig:calibrMethods}
\end{figure}


\subsection{Método} \label{ssc:classicalMethod}

A parte de la medición de la estabilidad del instrumento, es necesario obtener el funcionamiento de los TRMs de forma
individual, dado que el apuntamiento y la planitud de la señal dependen de la configuración de estos componentes. El uso de 
datos de telemetría (por ejemplo, temperaturas y tensiones de los TRMs) junto a una caracterización apropiada en tierra, 
solamente sirve para obtener información limitada del comportamiento de la antena \cite{Br2007}.

Una estrategia para mediciones individuales de cada TRM requiere que el resto de dichos componentes estén apagados. El problema
de esta estrategia es que, al tener parte de la antena apagada, no resulta un método representativo al modo de funcionamiento 
nominal (toda la antena operativa). Para solventar esta problemática y calibrar todos los TRMs a la vez, al menos en 
transmisión o recepción, se hace uso de los defasadores para armar una codificación diferente para cada TRM. En particular, 
por ejemplo, se pueden utilizar los códigos walsh \cite{WalshCode} defasando cada TRM en $\pm90^{\circ}$, siguiendo una 
determinada secuencia $c_{mn}(t)$.

\begin{figure}
 \centering
 \includegraphics[width=8cm]{gfx/superposition_signals_classic.png}
 \caption{Superposición de señales de todos los TRMs. Cada señal tiene su propia secuencia de código aplicada entre pulsos \cite{Br2007}.}
 \label{fig:sup_sign_classic}
\end{figure}

Por lo tanto, la fase de salida de cada TRM es la fase configurada $\varphi_{mn}$ sumada al defasaje del código de
$90^{\circ}$. Consecuentemente, la superposición de todas las ganancias de los TRMs, $a_{mn}$, y fases, $\varphi_{mn}$,
es obtenida en el puerto de recepción de la RFDN, $s_c(t)$ como se muestra en la figura \ref{fig:sup_sign_classic}.

\begin{equation}
	s_c(t) = \sum_{m=0}^{M-1}\sum_{n=0}^{N-1}c_{mn}\cdot a_{mn}e^{j\varphi{mn}} + n_{mn}
\end{equation}

Donde $n_{mn}$ es el ruido inherente que hay en las mediciones de cada TRM. Para decodificar y obtener la ganancia
$\tilde{a}_{mn}$ y fase estimada $\tilde\varphi_{mn}$ de algún TRM, la señal compuesta $s_c$ es correlacionada con
la secuencia del módulo deseado. Con esta correlación, la modulación de la secuencia se elimina dando como resultado
la ganancia estimada.

\begin{equation}
\begin{aligned}
	\tilde{x}_{mn} &= s_c \otimes c^*_{mn} \\
	\tilde{x}_{mn} = \int s_c(t) &\cdot c^*_{mn}(t) dt = \tilde{a}_{mn}e^{j\tilde{\varphi}_{mn}} \\
\end{aligned}
\label{eq:classic_correlation}
\end{equation}

En general el código de calibración utilizado es el código walsh \cite{WalshCode}. Dicho código deriva de las matrices de 
Hadamard (ver apéndice \ref{AppendixB}); dada sus propiedades de ortogonalidad, cada código, o fila, es unívocamente 
distinguible del resto. Para minimizar la cantidad de mediciones, el largo del código ($l$) debe ser lo más corto positble. 
El número de TRMs de la antena es el determinante de la cota inferior.

\begin{equation}
	l = 2^i \ge N \cdot M
\end{equation}

Siendo $N$ la cantidad de filas y $M$ la cantidad de paneles, o columnas, que tiene el conjunto de antena. No es necesario que
se calibren todos los TRMs de una. Hay tres estrategias que se utilizan con este método de calibración para obtener
distintos niveles de granularidad de mediciones a saber.

\begin{itemize}
	\item \textbf{Nivel módulo:} Este nivel es el que utiliza los códigos más largos, dado que se calibran todos los módulos que
		posee la antena en una polarización determinada ($l = 2^i \ge N \cdot M$).
	\item \textbf{Nivel panel:} En este nivel se utiliza el mismo código para todos los TRMs que son de un mismo panel,
		logrando así, decrecer el largo del código ($l = 2^i \ge M$).
	\item \textbf{Nivel fila:} En este nivel se utiliza el mismo código para todos los TRMs que son de una misma fila,
		logrando así, decrecer el largo del código ($l = 2^i \ge N$).
\end{itemize}

Este método a nivel panel y fila sirve para la caracterización del la configuración del apuntamiento de antena \cite{Br2007}.

\subsection{Problemas y limitaciones}

La calibración clásica fue diseñada para detectar y localizar fallas en los módulos TR permitiendo visualizar como cambia la
fase y atenuación de cada uno a medida que se los comanda a traves de una electrónica central. Como contracara, adolece de
los siguientes problemas:

\begin{itemize}
	\item Caracterización previa de los elementos de antena: Para poder conocer la ganancia del lazo de transmisión o recepción
es necesario conocer la potencia de la señal de calibración, para sustraerla del resultado obtenido. El lazo de calibración
donde se realiza esta medición es el que se muestra en la imagen \ref{fig:classic_cal_scheme}, llamado \textbf{P3}. En esta
medición no se puede determinar cuanto atenúa el lazo, por lo tanto se opta por caracterizar en tierra dichos componentes
para las frecuencias y temperaturas de trabajo.
	\begin{itemize}
		\item Costo de materiales: Materiales caracterizados en temperatura valen el doble que los no caracterizados, 
			dado que no es un trabajo sencillo y requiere el uso de cámaras de termovacío para realizar los ensayos.
		\item Costo de recursos: La campaña de caracterización puede durar meses, con equipos de trabajo dedicados, lo cual implica
			un gasto de dinero muy importante.
		\item Comportamiento de materiales fin de vida: Como el material envejece, cambia sus propiedades, por lo tanto las mediciones realizadas
			en la campaña de caracterización dejan de tener validez.
	\end{itemize}
	\item Complejidad del hardware de la antena: Como se calibra solo una parte de la antena por vez, transmisión o recepción en
		una u otra polarización (H o V), es necesario que el lazo de calibración esté compuesto por la parte de la antena a
		calibrar junto a hardware dedicado (cables y switches) a esta tarea. Logrando así, no solo que la construcción de la antena
		sea más compleja y que se tengan que caracterizar más componentes, sino también que el defasaje y atenuación que este
		hardware dedicado posee es atribuido a los TRMs, añadiendo así más error en la medición.
	\begin{itemize}
		\item Acoplamiento: Como hay hardware agregado, el diseño tiene que se más complicado dado que se incrementan los posibles
			acoplamientos entre componentes.
	\end{itemize}
	\item Inestabilidad del generador: Este método es sumamente susceptible a las variaciones de fase y potencia del generador
		entre pulsos de calibración. Por lo tanto, es de vital importancia armar un generador que sea sumamente estable.
	\item Sistema incompleto: Como hay componentes que están fuera de los lazos de calibración, con el método nunca se puede
		determinar de forma totalmente correcta la ganacia de la antena en cualquiera de sus modos.
	\item A la hora de elegir la longitud del código Walsh, es importante que siempre haya un elemento radiante virtual en la
		antena para evitar la primer columna de la matriz de códigos. En caso contrario el primer TRM siempre tendrá un error en
		la estimación de su ganancia \cite{Wang2010}.
	\item Pérdida de ortogonalidad en el código utilizado por comportamiento de defasadores: Si el defasador no tiene un
		comportamiento exactamente igual al configurado a la hora de realizar la calibración, los códigos walsh pierden
		ortogonalidad y, por ende, se logra obtener una mala estimación de los valores de ganancia de la antena.
    \item Compensacion de temperatura en elementos: Se debe mantener la temperatura de los elementos en los valores
        caracterizados para garantizar que las calibraciones sirven.
    \item Es completamente dependiente del modelo y análisis térmico. Por ejemplo, se asume que el comportamiento de los
        elementos radiantes de la antena no varían con la temperatura.
\end{itemize}
\end{comment}
