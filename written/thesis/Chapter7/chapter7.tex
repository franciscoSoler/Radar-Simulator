\chapter{Líneas Futuras} \label{ch:futureWork}

% **************************** Define Graphics Path **************************
\ifpdf
    \graphicspath{{Chapter6/Figs/Raster/}{Chapter6/Figs/PDF/}{Chapter6/Figs/}}
\else
    \graphicspath{{Chapter6/Figs/Vector/}{Chapter6/Figs/}}
\fi

A continuación se listan diversas modificaciones que se pueden aplicar tanto como para disminuir las incertidumbres asociadas como para realizar investigaciones de nuevas aplicaciones.

\begin{itemize}
    \item Para evitar la utilización de la misma señal para determinar tanto los parámetros S del blanco como la distancia entre el radar y el mismo, se puede disponer de un medidor de distancia láser externo.
    \item Se pueden aplicar distintas estrategias a la aplicada para suprimir el clutter para la medición.
    \item Se puede mejorar el procesador de señales para el caso que hayan más de un blanco iluminado a la vez.
    \item Se pueden cambiar las antenas para disminuir el crosstalk entre ellas y para obtener una mejor calidad de respuesta de sus parámetros S.
    \item Se puede incluir un sistema de calibración para calibrar la señal transmitida y recibida ante el cambio del comportamiento de los distintos subsistemas que componen el radar.
    \item Para mejorar el rango de señal de la computadora, se puede agregar otro amplificador luego del pasa bajos para amplificar aún más la señal antes de ser digitalizada.
    \item Para realizar mediciones a mayores distancias se puede utilizar un ADC con una mayor frecuencia de muestreo y utilizar un filtro pasa bajos de mayor frecuencia de corte.
    \item Se puede aplicar un sistema para mejorar las alinealidades de la señal transmisora.
    \item Se puede realizar un estudio para determinar la mejor modulación de la señal a transmitir para la obtención del parámetros de dispersión del blanco iluminado.
    \item Se puede agregar un switch RF para poder elegir de forma automática las polarizaciones transmisoras y receptoras de forma automática.
    \item Se puede utilizar un conjunto de antenas para estudiar como se comporta este radar como un sistema SAR.
\end{itemize}